\chapter{Evaluation}
    \section{Angular Sensor}
        Before investigating the dependence of the angle and the timer ticks, it is needed to determine the mean value of the timer
        ticks for each angle. The mean values are computed taking 20 samples each.\par
        \begin{table}[h]
            \caption[Mean value of timer ticks per angular displacement]{Mean value of timer ticks per angular displacement}
            \centering
            \begin{tabular}{@{}S[table-format=+2]S[table-format=3.2]cS[table-format=+2]S[table-format=+3.2]@{}}
                \toprule
                \varphi / \si[]{\degree}            &\bar{n}        &\hspace{20mm}  &\varphi / \si[]{\degree}   &\bar{n}\\
                \midrule
                +90                                 &601.34         &               &-10                        &-55.33\\
                +80                                 &549.63         &               &-20                        &-126.04\\
                +70                                 &482.26         &               &-30                        &-195.80\\
                +60                                 &421.93         &               &-40                        &-262.31\\
                +50                                 &353.10         &               &-50                        &-331.20\\
                +40                                 &288.13         &               &-60                        &-396.87\\
                +30                                 &220.58         &               &-70                        &-459.46\\
                +20                                 &147.85         &               &-80                        &-474.44\\
                +10                                 &83.57          &               &-90                        &-499.79\\
                +0                                  &14.36          &               &                           &\\
                \bottomrule
            \end{tabular}
            \label{tab:mean ticks per angle}
        \end{table}
        %
        In \cref{fig:angular-sensor} the timer ticks \(n\) are plotted as a function of the angle \(\varphi\).
        %
        \begin{figure}[H]
            \centering
            \includegraphics[width=1\linewidth]{"messdaten/Angular Sensor"}
            \caption[Curve of the angular sensor]{Curve of the angular sensor}
            \label{fig:angular-sensor}
        \end{figure}
        %
        The sensitivity corresponds to the slope of the curve and the offset to the intercept.\par
        The computed fit is
        %
        \begin{equation}
            n(\varphi)=\SI{373.60}{rad^{-1}}\cdot\varphi +19.03
        \end{equation}
        %
        which gives values for the sensitivity \( s \) and the offset \( o \) of
        %
        \begin{align}
            s&=\SI{374}{rad^{-1}} \pm \SI{1}{rad^{-1}}\\
            o&=19 \pm 1
        \end{align}
        %
        \begin{figure}
            \centering
            \begin{framed}
                \textbf{Cihans Comparison}
            \end{framed}
        \end{figure}

    \section{Torsional Pendulum}
    %
        \subsection{Natural Angular Frequency And Damping Coefficient}
            \Cref{fig:damped-oscillation-100mA} shows the oscillation with a coil current of \(\SI{100}{mA}\). The respective plots
            for coil currents ranging from \(\SI{200-500}{mA}\) and the coil current turned off can be seen in \cref{fig:damped-oscillations}.
            Deflection is plotted versus time. The period time for each curve is determined by reading a few values and building the mean.\par
            %
            \begin{figure}[H]%
                \centering
                \includegraphics[width=1\linewidth]{"messdaten/Damped Oscillation (100mA)"}
                \caption[Course of the tick rate at \(I_{c}=\SI{100}{mA}\)]{Plot of the tick rate over time with a coil current of \SI{100}{mA}.}
                \label{fig:damped-oscillation-100mA}
            \end{figure}
            %
            The values are as follows:\par
            \begin{table}[h]
                \centering
                \caption[Period times for the damped oscillations]{}
                \begin{tabular}{@{}llll@{}}
                    \toprule
                    \( T_0 \)       &$\SI{9.7}{s} \pm \SI{0.5}{s}$    &\hspace{10mm}\(T_{300}\)      &$\SI{9.8}{s} \pm \SI{0.5}{s}$\\
                    \( T_{100} \)   &$\SI{9.6}{s} \pm \SI{0.5}{s}$    &\hspace{10mm}\(T_{400}\)      &$\SI{10.0}{s} \pm \SI{0.5}{s}$\\
                    \(T_{200}\)     &$\SI{9.6}{s} \pm \SI{0.5}{s}$    &\hspace{10mm}\(T_{500}\)      &$\SI{10.4}{s} \pm \SI{0.5}{s}$\\
                    \bottomrule
                \end{tabular}
                \label{tab:period_times_damped_oscillations}
            \end{table}
            %
            Thus, the damped natural frequencies are calculated using \(\omega=\frac{2\pi}{T}\):\par
            \begin{table}[h]
                \centering
                \caption[Damped natural frequencies]{}
                \begin{tabular}{@{}llll@{}}
                    \toprule
                    $\omega_{d,0}$      &$\SI{0.65}{s^{-1}} \pm \SI{0.03}{s^{-1}}$  &\hspace{10mm}$\omega_{d,300}$  &$\SI{0.64}{s^{-1}} \pm \SI{0.03}{s^{-1}}$\\
                    $\omega_{d,100}$    &$\SI{0.65}{s^{-1}} \pm \SI{0.03}{s^{-1}}$  &\hspace{10mm}$\omega_{d,400}$  &$\SI{0.63}{s^{-1}} \pm \SI{0.03}{s^{-1}}$\\
                    $\omega_{d,200}$    &$\SI{0.64}{s^{-1}} \pm \SI{0.03}{s^{-1}}$  &\hspace{10mm}$\omega_{d,500}$  &$\SI{0.60}{s^{-1}} \pm \SI{0.03}{s^{-1}}$\\
                    \bottomrule
                \end{tabular}
                \label{tab:damped_natural_frequencies}
            \end{table}
            %
            The deviations of the period time are reading errors. The ones of the damped natural frequency can be calculated with\par
            %
            \begin{align}
                \Delta \omega_d &=\left|\frac{\partial \omega_d}{\partial T}\right| \cdot \Delta T \nonumber \\
                                &=\frac{2\pi}{T^2}\cdot \Delta T
            \end{align}
            %
            for \(\omega_{d,0}\) e.g.:\par
            %
            \begin{align}
                \Delta\omega_{d,0}  &=\frac{2\pi}{(\SI{9.7}{s})^2}\cdot \SI{0.5}{s} \nonumber \\
                                    &=\SI{0.03}{s^{-1}}
            \end{align}
            %
            The damping coefficient can be determined by way of\par
            %
            \begin{align}
                \varphi_1 \cdot e^{-\delta t_0} &=\varphi_0 \cdot e^{-\delta t_1} \nonumber \\%   &&\Bigg|\cdot \frac{e^{\delta t_1}}{\varphi_1} \nonumber\\
                e^{\delta(t_1-t_0)}             &=\frac{\varphi_0}{\varphi_1} \nonumber \\%       &&\Bigg|\ln \nonumber\\
                \delta(t_1-t_0)                 &=\ln{\frac{\varphi_0}{\varphi_1}} \nonumber \\%  &&\Bigg|\cdot \frac{1}{(t_1-t_0)} \nonumber\\
                \delta                          &=(t_1-t_0)^{-1}\cdot\ln{\frac{\varphi_0}{\varphi_1}}
            \end{align}
            %
            The times and angles have been read and \(\delta\) has been calculated individually.\par
            This gives us the following mean values:\par
            %
            \begin{table}[h]
                \centering
                \caption[Mean values of the dampening coefficient]{}
                \begin{tabular}{@{}llll@{}}
                    \toprule
                    $\delta_0$        &$\SI{0.002}{s^{-1}} \pm \SI{0.001}{s^{-1}}$  &\hspace{10mm}$\delta_{300}$ &$\SI{0.08}{s^{-1}} \pm \SI{0.01}{s^{-1}}$\\
                    $\delta_{100}$    &$\SI{0.008}{s^{-1}} \pm \SI{0.002}{s^{-1}}$  &\hspace{10mm}$\delta_{400}$ &$\SI{0.12}{s^{-1}} \pm \SI{0.01}{s^{-1}}$\\
                    $\delta_{200}$    &$\SI{0.03}{s^{-1}} \pm \SI{0.01}{s^{-1}}$    &\hspace{10mm}$\delta_{500}$ &$\SI{0.22}{s^{-1}} \pm \SI{0.01}{s^{-1}}$\\
                    \bottomrule
                \end{tabular}
                \label{tab:dampening_coefficients}
            \end{table}
            %
            For the deviation, the standard deviation is used. Otherwise, it can also be determined by means of the partial derivation:\par
            %
            \begin{align}
                \Delta\delta    &= \left| \frac{\partial\delta}{\partial t_1} \right| \cdot \Delta t_1 + \left| \frac{\partial\delta}{\partial t_0} \right| \cdot \Delta t_0 + \left| \frac{\partial\delta}{\partial \varphi_0} \right| \cdot \Delta \varphi_0 + \left| \frac{\partial\delta}{\partial \varphi_1} \right| \cdot \Delta \varphi_1 \nonumber\\
                                &= \frac{1}{(t_1-t_0)^2} \cdot \ln{\frac{\varphi_0}{\varphi_1}} \cdot (\Delta t_0 + \Delta t_1) + \frac{1}{(t_1-t_0)} \cdot \left( \frac{1}{\varphi_0} \cdot \Delta\varphi_0 + \frac{1}{\varphi_1} \cdot \Delta\varphi_1 \right)
            \end{align}%
            The natural angular frequencies are calculated via $ \omega_0=\sqrt{\omega_d^2+\delta^2} $ as\par
            %
            \begin{table}[h]
                \centering
                \caption[Angular natural frequencies]{}
                \begin{tabular}{@{}llll@{}}
                    \toprule
                    $\omega_{0,0}$      &$\SI{0.65}{s^{-1}} \pm \SI{0.03}{s^{-1}}$  &\hspace{10mm}$\omega_{0,300}$   &$\SI{0.65}{s^{-1}} \pm \SI{0.03}{s^{-1}}$\\
                    $\omega_{0,100}$    &$\SI{0.65}{s^{-1}} \pm \SI{0.03}{s^{-1}}$  &\hspace{10mm}$\omega_{0,400}$   &$\SI{0.64}{s^{-1}} \pm \SI{0.03}{s^{-1}}$\\
                    $\omega_{0,200}$    &$\SI{0.64}{s^{-1}} \pm \SI{0.03}{s^{-1}}$  &\hspace{10mm}$\omega_{0,500}$   &$\SI{0.68}{s^{-1}} \pm \SI{0.03}{s^{-1}}$\\
                    \bottomrule
                \end{tabular}
                \label{tab:angular_natural_frequencies}
            \end{table}
            %
            and their deviations via\par
            %
            \begin{align}
                \Delta\omega_0  &=\left| \frac{\partial\omega_0}{\partial\omega_d} \right| \cdot \Delta\omega_0 + \left| \frac{\partial\omega_0}{\partial\delta} \right| \cdot \Delta\delta \nonumber\\
                                &=\frac{\omega_d}{\sqrt{\omega_d^2+\delta^2}} \cdot \Delta\omega_d + \frac{\delta}{\sqrt{\omega_d^2+\delta^2}} \cdot \Delta\delta
            \end{align}
            %
            for \(\omega_{0,0}\) e.g. as\par
            %
            \begin{align*}
                \Delta\omega_{0,0}  &=\frac{\SI{0.65}{s^{-1}}}{\sqrt{\SI{0.65}{s^{-1}}+\SI{0.002}{s^{-1}}}} \cdot \SI{0.03}{s^{-1}} + \frac{\SI{0.002}{s^{-1}}}{\sqrt{\SI{0.65}{s^{-1}}+\SI{0.002}{s^{-1}}}} \cdot \SI{0.001}{s^{-1}}\\
                                    &=\SI{0.03}{s^{-1}}+\SI{3}{\cdot10^{-6}s^{-1}}\\
                                    &\approx\SI{0.03}{s^{-1}}
            \end{align*}
            %
            In \cref{fig:damping-coefficient} the damping coefficient is plotted versus coil current. For finding a formula
            \(\delta(I)\) SciDAVis did an exponential fit \(\delta_e(I)\) (red curve) and a fourth degree polynomial fit
            \(\delta_p(I)\) (blue curve):\par
            %
            \begin{align}
                \delta_e(I)=&\SI{0.013}{s^{-1}}\cdot e^{\SI{0.0052}{A^{-1}}\cdot I}\\
                \delta_p(I)=&\SI{1.8}{\cdot 10^{-3}s^{-1}} - \SI{1.5}{\cdot 10^{-5}s^{-1}A^{-1}}\cdot I + \SI{6.4}{\cdot 10^{-7}s^{-1}A^{-2}}\cdot I^2 \nonumber\\
                            &+ \SI{1.2}{\cdot 10^{-9}s^{-1}A^{-3}}\cdot I^3 + \SI{2.2}{\cdot 10^{-12}s^{-1}A^{-4}}\cdot I^4
            \end{align}
            %
            \begin{figure}
                \centering
                \includegraphics[width=1\linewidth]{"messdaten/Damping Coefficient"}
                \caption[Damping coefficient depending on the coil current]{Damping coefficient depending on the coil current}
                \label{fig:damping-coefficient}
            \end{figure}
            %
            As it can be seen, the polynomial fit is a better approximation than the exponential. However, the polynomial curve
            is only a good approach in the area of \(0 \leq I \leq \SI{500}{mA}\). At higher currents the curve has a maximum
            and goes lower again, which is illogical in the physical sense. Therefore, the exponential curve is a qualitatively
            better description of the dependency.\par\medskip
            Finally, the aperiodic case is considered and plotted in \cref{fig:damped-oscillation-aperiodic}.\par
            %
            \begin{figure}
                \centering
                \includegraphics[width=1\linewidth]{"messdaten/Damped Oscillation (aperiodic)"}
                \caption[Aperiodic damped oscillation at \(I_c=\SI{1.3}{A}\)]{Aperiodic damped oscillation at \(I_c=\SI{1.3}{A}\)}
                \label{fig:damped-oscillation-aperiodic}
            \end{figure}
            %
        \subsection{Rotational Inertia}
            To determine the unknown rotational inertia of the pendulum, the dimensions of each of the three rods must first be measured
            as it can be seen in \cref{rod dimensions}.\par
            %
            \begin{table}[H]
                \centering
                \caption{rod dimensions}
                \label{rod dimensions}
                \begin{tabular}{|c|c|c|c|}
                    \hline
                    i & $ m_{Ri} / \SI{}{g} $ & $ D_{Ri} / \SI{}{mm} $ & $ l_{Ri} / \SI{}{mm} $ \\
                    \hline
                    \hline
                    1 & 8.89 $\pm$ 0.01 & 5.95 $\pm$ 0.05 & 120 $\pm$ 0.05 \\
                    \hline
                    2 & 26.73 $\pm$ 0.01 & 6.00 $\pm$ 0.05 & 120 $\pm$ 0.05 \\
                    \hline
                    3 & 56.38 $\pm$ 0.01 & 6.00 $\pm$ 0.05 & 240 $\pm$ 0.5 \\
                    \hline
                \end{tabular}
            \end{table}
            %
            Obviously, the diameters all are much smaller than the lengths. Hence, the rotational rod inertia can be calculated
            by means of \cref{eq:rotationalIntertia_of_Cyl alternate}.\par
            The following values are obtained:\par
            %
            \begin{align}
                J_{R1}  &=\SI{10670}{g\cdot mm^2} \pm \SI{21}{g\cdot mm^2}\\
                J_{R2}  &=\SI{32080}{g\cdot mm^2} \pm \SI{39}{g\cdot mm^2}\\
                J_{R3}  &=\SI{270600}{g\cdot mm^2} \pm \SI{1200}{g\cdot mm^2}
            \end{align}
            %
            The deviations of the rod dimensions are reading errors of the scales. The deviations of the rod inertia can be determined by way of:\par
            %
            \begin{align}
                \Delta J_{Ri}   &=\left| \frac{\partial J_{Ri}}{\partial m_{Ri}} \right| \cdot \Delta m_{Ri} + \left| \frac{\partial J_{Ri}}{\partial l_{Ri}} \right| \cdot \Delta l_{Ri} \nonumber\\
                                &=\frac{1}{12}\cdot l_{Ri}^2\cdot \Delta m_{Ri} + \frac{1}{6}\cdot m_{Ri}\cdot l_{Ri}\cdot \Delta l_{Ri}
            \end{align}
            %
            For \(J_{R2}\) it is e.g.:\par
            %
            \begin{align*}
                \Delta J_{R2}   &=\frac{1}{12}\cdot (\SI{120}{mm})^2\cdot \SI{0.01}{g} + \frac{1}{6}\cdot \SI{26.73}{g}\cdot \SI{120}{mm}\cdot \SI{0.05}{mm}\\
                                &=\SI{12}{g\cdot mm^2}+\SI{26.73}{g\cdot mm^2}\\
                                &=\SI{38.73}{g\cdot mm^2} \approx \SI{39}{g\cdot mm^2}
            \end{align*}
            %
            The mean values of the period times read with reading errors are as follows:\par
            %
            \begin{align}
                T_{P+R1} &=\SI{9.7}{s} \pm \SI{0.3}{s} \\
                T_{P+R2} &=\SI{9.8}{s} \pm \SI{0.3}{s} \\
                T_{P+R3} &=\SI{11.3}{s} \pm \SI{0.3}{s} \\
                T_P &=\SI{9.7}{s} \pm \SI{0.3}{s} \qquad (\text{without rod})
            \end{align}
            %
            With \cref{eq:inertia} the unknown pendulum inertia are calculated as:
            %
            \begin{align}
                J_{P1} &= \text{(error)} \\
                J_{P2} &= \SI{1286000}{g \cdot mm^2} \pm \SI{6000}{g \cdot mm^2} \\
                J_{P3} &= \SI{760000}{g \cdot mm^2} \pm \SI{34000}{g \cdot mm^2}
            \end{align}
            %
            The uncertainties are determined as follows:
            %
            \begin{align}
                \Delta J_P  &= \left| \frac{\partial J_P}{\partial J_R} \right| \cdot \Delta J_R + \left| \frac{\partial J_P}{\partial T_{P+R}} \right| \cdot \Delta T_{P+R} + \left| \frac{\partial J_P}{\partial T_P} \right| \cdot \Delta T_P \nonumber \\
                            &= \frac{1}{\left( \frac{T_{P+R}}{T_P} \right)^2-1} \cdot \Delta J_R + \frac{2 \cdot J_R \cdot T_{P+R}}{T_P^2 \cdot \left( \frac{T_{P+R}^2}{T_P^2}+1 \right)} \cdot \Delta T_{P+R} + \frac{2 \cdot J_R \cdot T_{P+R}^2}{T_P^3 \cdot \left( \frac{T_{P+R}^2}{T_P^2}+1 \right)} \cdot \Delta T_P \qquad \Big| \Delta T_{P+R} = \Delta T_P \nonumber \\
                            &= \frac{1}{\left( \frac{T_{P+R}}{T_P} \right)^2-1} \cdot \Delta J_R + \left( \frac{2 \cdot J_R \cdot T_{P+R}}{T_P^2 \cdot \left( \frac{T_{P+R}^2}{T_P^2}+1 \right)} + \frac{2 \cdot J_R \cdot T_{P+R}^2}{T_P^3 \cdot \left( \frac{T_{P+R}^2}{T_P^2}+1 \right)} \right) \cdot \Delta T_P
            \end{align}
            %
            For \(J_{P3}\) it results:
            %
            \begin{align*}
                \Delta J_{P3}   &= \frac{1}{\left( \frac{\SI{11.3}{s}}{\SI{9.7}{s}} \right)^2-1} \cdot \SI{1200}{g \cdot mm^2} + \left( \frac{2 \cdot \SI{270600}{g \cdot mm^2} \cdot \SI{11.3}{s}}{(\SI{9.7}{s})^2 \cdot \left( \frac{(\SI{11.3}{s})^2}{(\SI{9.7}{s})^2}+1 \right)} + \frac{2 \cdot \SI{270600}{g \cdot mm^2} \cdot (\SI{11.3}{s})^2}{(\SI{9.7}{s})^3 \cdot \left( \frac{(\SI{11.3}{s})^2}{(\SI{9.7}{s})^2}+1 \right)} \right) \cdot \SI{0.3}{s} \\
                                &= \SI{3360}{g \cdot mm^2} + \SI{30981}{g \cdot mm^2} \\
                                &= \SI{34341}{g \cdot mm^2} \approx \SI{34000}{g \cdot mm^2}
            \end{align*}
            %
            The calculable values for the pendulum inertia are around \(\SI{10^6}{g \cdot mm^2}\). The short aluminum rod
            gives no value for the pendulum inertia, since it shows no noticeable differences in period time. So it seems
            that the aluminum rod does not change the inertia of the pendulum at all. Furthermore, the two values obtained
            have relatively large deviations from one another. It is probably due to the avoidable but also less avoidable
            measurement errors. The latter are external influences, e.g. the vibration of the pendulum caused by table
            tremors or shaking hands when deflecting the pendulum. On the other hand, reading the period time could be
            more accurate. As it can be seen in the calculation of the uncertainties, the period time deviation is more
            perceptible than the rod inertia deviation.
            