\chapter{Execution}
    \section{Angular Sensor}
        The current through the electromagnet, which operates the eddy current brake, is set to \(I = \SI{1}{A}\). With the
        zero adjustment (No. 3 in \cref{fig:setup_total}), the copper disk is set to a deflection of \(\varphi = \SI{-90}{\degree}\).
        The recording using \textsc{RealTerm} is started. Some values are captured and saved. The deflection is decreased by \SI{10}{\degree}
        and a new recording is started and saved. This process is repeated in increments of \SI{10}{\degree} until a deflection of
        \(\varphi = \SI{+90}{\degree}\) is reached. At the end, the current through the electromagnet is set back to \(I = \SI{0}{A}\).
%
    \section{Torsional Pendulum}
        \subsection{Natural Angular Frequency and Damping Coefficient}
            The deflection of the pendulum is set back to \(\varphi = \SI{0}{\degree}\) by zero adjustment. Seven series of measurements
            are recorded and saved. Each measurement series runs for \SI{300}{s}. Each time a different current is set, as shown
            in \cref{tab:currents}. At each measurement, the disk is carefully deflected to \(\varphi = \SI{75}{\degree}\) and
            held. At the moment the reset button on the microcontroller is pressed, the pendulum is released while avoiding
            unnecessary oscillations. At higher currents, the oscillation ends before the completion of the \SI{300}{s}. In these cases, the
            capturing is ended earlier.
        \begin{table}[H]
            \centering
            \caption[Currents set]{Currents set in each measurement.}
            \label{tab:currents}
            \begin{tabular}{@{}ll@{}}
                \toprule
                no.     & $I$\\
                \midrule
                1       & \SI{0}{A}\\
                2       & \SI{0.1}{A}\\
                3       & \SI{0.2}{A}\\
                4       & \SI{0.3}{A}\\
                5       & \SI{0.4}{A}\\
                6       & \SI{0.5}{A}\\
                7       & \SI{1.3}{A}\\
                \bottomrule
            \end{tabular}
        \end{table}
        At the end, the current through the electromagnet is set back to \(I = \SI{0}{A}\).
        %
        \subsection{Rotational Inertia}
            The diameter, length, and mass of the three given rods are measured and the values, including their uncertainty, are noted.
            For this purpose, a measuring caliper, a steel ruler and the balance provided in the laboratory rooms are used. The current
            is set to \(I = \SI{100}{mA}\). The first rod is inserted into the cross bore and carefully screwed tight. The hole is
            located between the copper plate and the angular sensor. An already mounted rod is shown in \cref{fig:setup_detailed}. When
            mounting, the rod is positioned as centrally as possible in the socket. The disk is carefully deflected to
            \(\varphi = \SI{75}{\degree}\) and held. At the moment the reset button on the microcontroller is pressed, the pendulum
            is released avoiding unnecessary oscillations. The measurement ends after the pendulum has come to rest. The data is saved
            and the measurement is repeated for the other rods.